\documentclass[a4paper, 12pt]{article}

\input{~/Desktop/Studia/LaTeX/setup_eng.tex}
\author{Wojciech Orlowski}
\title{Introduction to spintronics – spin transistor in \texttt{Kwant} package}

\begin{document}
\maketitle

\section*{Introduction}

Using \texttt{Kwant} package few important tranpsort properties including spin interaction has been calculated.
In order to include spin interaction in calculations most of values (hoppings and onsite) are represented as 2D arrays.

\section*{Spin precession in external magnetic field}

System used in calculations was simple nanowire with two leads (fig. \ref{fig:ex1_system}).

\begin{figure}[h]
    \begin{center}
        \includegraphics[width=0.5\textwidth]{../plots/ex1_system.pdf}
    \end{center}
    \caption{System used in calculations}
    \label{fig:ex1_system}
\end{figure}

As a check for correct build of system firstly dispersion relation without magnetic field has been calculated (fig. \ref{fig:ex1_disp}).

\begin{figure}[H]
    \begin{center}
        \includegraphics[width=0.30\textwidth]{../plots/ex1_disp.pdf}
    \end{center}
    \caption{Dispersion relation in system without magnetic field.}
    \label{fig:ex1_disp}
\end{figure}

Next magnetic field has been applied in directions $x$, $y$ and $z$.
For all directions dispersion relation has been calculated (fig. \ref{fig:ex1_disp_magn}).

\begin{figure}[h]
    \begin{center}
        \begin{subfigure}{0.30\textwidth}
            \includegraphics[width=0.95\textwidth]{../plots/ex1_disp_bx_1.pdf}
            \caption{}
        \end{subfigure}
        \begin{subfigure}{0.30\textwidth}
            \includegraphics[width=0.95\textwidth]{../plots/ex1_disp_by_1.pdf}
            \caption{}
        \end{subfigure}
        \begin{subfigure}{0.30\textwidth}
            \includegraphics[width=0.95\textwidth]{../plots/ex1_disp_bz_1.pdf}
            \caption{}
        \end{subfigure}
    \end{center}
    \caption{Dispersion relation after applyingmagnetic field with amplitude 1T in (a) $x$, (b) $y$, (c) $z$ direction.}
    \label{fig:ex1_disp_magn}
\end{figure}

Observed Zeeman splitting does not depend on the direction of magnetic field.

For the next step conductance as a function of incident electron energy has been calculated (fig. \ref{fig:ex1_cond}).
Conductance has been calculated with constant magnetic field in $z$ direction with amplitude 1T.

\begin{figure}[h]
    \begin{center}
        \includegraphics[width=0.6\textwidth]{../plots/ex1_conductance_bz_1.pdf}
    \end{center}
    \caption{Conductance}
    \label{fig:ex1_cond}
\end{figure}

\newpage 
Additional magnetic field has been introduced.
In the region from $0.2 \cdot L$ to $0.8 \cdot L$ additional magnetic field has been applied in $y$ direction.
Tranmission coefficients including spin rotation has beend calculated as a function of $B_y$ field (fig. \ref{fig:ex1_trans}).
Value of $B_z$ has been changed to 0.1 T.
Energy of electron has been equal to 0.005 eV.

\begin{figure}[h]
    \begin{center}
        \includegraphics[width=0.6\textwidth]{../plots/ex1_transmittances.pdf}
    \end{center}
    \caption{Transmission coefficient for different incident electrons as a function external magnetic field in $y$ direction.}
    \label{fig:ex1_trans}
\end{figure}

For the same system $B_y = 0.6$ has been selected.
<aps of charge density for spin up and spin down has been calculated (fig. \ref{fig:ex1_density_maps}).

\begin{figure}[h]
    \begin{center}
        \begin{subfigure}{0.49\textwidth}
            \includegraphics[width=0.95\textwidth]{../plots/ex1_density_up.pdf}
            \caption{}
        \end{subfigure}
        \begin{subfigure}{0.49\textwidth}
            \includegraphics[width=0.95\textwidth]{../plots/ex1_density_down.pdf}
            \caption{}
        \end{subfigure}
    \end{center}
    \caption{Charge density for (a) spin up, (b) spin down.}
    \label{fig:ex1_density_maps}
\end{figure}

For the same system maps of spin density has been calculated (fig. \ref{fig:ex1_spin_maps})

\begin{figure}[H]
    \begin{center}
        \begin{subfigure}{0.49\textwidth}
            \includegraphics[width=0.95\textwidth]{../plots/ex1_density_x.pdf}
            \caption{}
        \end{subfigure}
        \begin{subfigure}{0.49\textwidth}
            \includegraphics[width=0.95\textwidth]{../plots/ex1_density_y.pdf}
            \caption{}
        \end{subfigure}
        \\
        \begin{subfigure}{0.49\textwidth}
            \includegraphics[width=0.95\textwidth]{../plots/ex1_density_z.pdf}
            \caption{}
        \end{subfigure}
    \end{center}
    \caption{Spin density for (a) $\sigma_x$, (b) $\sigma_y$, (c) $\sigma_z$.}
    \label{fig:ex1_spin_maps}
\end{figure}

It has been shown that spin of electron can be manipulated using external magnetic field.

\section*{Spin transistor based on spin-orbit cupling}

Instead of using magnetic field in transitor (which would make device bigger) we can try to steer spin by Rashba SO interaction.
For new system $\alpha$ parameter controls transport properties. 
This parameter is controlled by external electric field (for instance induced by gate electrodes).

Firstly dispersion relation has been calculated assuming that SO interaction is present in device (fig. \ref{fig:ex2_disp}).

\begin{figure}[h]
    \begin{center}
        \includegraphics[width=0.35\textwidth]{../plots/ex2_disp.pdf}
    \end{center}
    \caption{Dispersion relation in system with SO coupling.}
    \label{fig:ex2_disp}
\end{figure}

Clearly visible is doubling number of bands.

Next conductance as a function of electron energy has been calculated (fig. \ref{fig:ex2_cond}).

\begin{figure}[h]
    \begin{center}
        \includegraphics[width=0.7\textwidth]{../plots/ex2_conductance.pdf}
    \end{center}
    \caption{Conductance as a function of incident electron energy.}
    \label{fig:ex2_cond}
\end{figure}

Characteristic steps in conductance are observable.

In following calculations SO interaction is present only from $0.2 \cdot L$ to $0.8 \cdot L$.
For this system transmission coefficients as a function of $\alpha$ parameter  has been calculated (fig. \ref{fig:ex2_trans}).
Transmission coefficients shows transmission to different subbands so also to different spins.

\begin{figure}[H]
    \begin{center}
        \includegraphics[width=0.7\textwidth]{../plots/ex2_transmittances.pdf}
    \end{center}
    \caption{Transmissions coefficienets as a function of $\alpha$ parameter.}
    \label{fig:ex2_trans}
\end{figure}

We can clearly see some values of $\alpha$ that respons to total transition from spin up to spin down and vice versa.

For the same system polarized onductance has been calculated as a function of $\alpha$.
It has been calculated for three different values of $P$ parameter (fig. \ref{fig:ex2_g_cond}).

\begin{figure}[h]
    \begin{center}
        \begin{subfigure}{0.49\textwidth}
            \includegraphics[width=\textwidth]{../plots/ex_2_cond_p=0.2.pdf}
            \caption{}
        \end{subfigure}
        \begin{subfigure}{0.49\textwidth}
            \includegraphics[width=\textwidth]{../plots/ex_2_cond_p=0.4.pdf}
            \caption{}
        \end{subfigure}
        \\
        \begin{subfigure}{0.49\textwidth}
            \includegraphics[width=\textwidth]{../plots/ex_2_cond_p=1.pdf}
            \caption{}
        \end{subfigure}
    \end{center}
    \caption{Polarized conductances as a function of $\alpha$ parameter for $P$ parameter equal to (a) 0.2, (b) 0.4, (c) 1}
    \label{fig:ex2_g_cond}
\end{figure}

For $\alpha$ value responding to the highest spin transition maps of charge density (fig. \ref{fig:ex2_density_maps}) and spin density (fig. \ref{fig:ex2_spin_map}) has been calculated.
This $\alpha$ parameter has been arbirtrary set to 0.018 eVnm.

\begin{figure}[H]
    \begin{center}
        \begin{subfigure}{0.49\textwidth}
            \includegraphics[width=0.95\textwidth]{../plots/ex2_density_up.pdf}
            \caption{}
        \end{subfigure}
        \begin{subfigure}{0.49\textwidth}
            \includegraphics[width=0.95\textwidth]{../plots/ex2_density_down.pdf}
            \caption{}
        \end{subfigure}
    \end{center}
    \caption{Charge density for (a) spin up, (b) spin down for incident electron energy 0.005 eV.}
    \label{fig:ex2_density_maps}
\end{figure}

Spin change is clearly visible nearly center of nanowire.

\begin{figure}[H]
    \begin{center}
        \begin{subfigure}{0.49\textwidth}
            \includegraphics[width=0.95\textwidth]{../plots/ex2_density_x.pdf}
            \caption{}
        \end{subfigure}
        \begin{subfigure}{0.49\textwidth}
            \includegraphics[width=0.95\textwidth]{../plots/ex2_density_y.pdf}
            \caption{}
        \end{subfigure}
        \\
        \begin{subfigure}{0.49\textwidth}
            \includegraphics[width=0.95\textwidth]{../plots/ex2_density_z.pdf}
            \caption{}
        \end{subfigure}
    \end{center}
    \caption{Spin density for (a) $\sigma_x$, (b) $\sigma_y$, (c) $\sigma_z$ for incident electron energy 0.005 eV.}
    \label{fig:ex2_spin_map}
\end{figure}

\section*{Summary}

\texttt{Kwant} again has been proven as a very useful tool in analyzing transport in nanoscopic system.
Analysis of spin dependent systems requires changing dimensionality of some data
Thankfully nice written \texttt{Kwant} API this change is not hard to implement.

\end{document}
