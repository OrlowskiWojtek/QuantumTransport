\documentclass[12pt, a4paper]{article}

\input{~/Desktop/Studia/LaTeX/setup_eng.tex}
\author{Wojciech Orłowski}
\title{Calculation of current-voltage characteristic of a resonant-tunneling \\
diode (RTD) and using the adiabatic approximation for simulation \\
of conductance quantization in quantum point contact (QPC)}

\begin{document}
\maketitle

\section*{Introduction}

Main goal of this laboratory is to implement transfer matrix method in many systems.
Firstly transmittance and reflectance through simple single barrier were calculated.
Then system was expanded by second barrier.

Our calculations were then used to calculate conductance using Tsu-Esaki formula.
Finally using adiabatic approximation we can calculate conductance throught quantum point contact.

All laboratory was written in \julia $\;$and source code is published at \href{https://github.com/OrlowskiWojtek/QuantumTransport/tree/2bb1eba9f6fcc11de530fe850330eb98d213b3ef/lab2}{GitHub} repository.

\newpage

\section*{Task 1}

In first task transmittance and reflectance throught simple potential barrier was calculated.
The barrier consists of Al0.3Ga0.7As and its height is equal to 0.27 eV.
Effective mass of electron inside GaAs is equal to 0.067, but inside barrier 0.0879.
We can compare transmittance and reflectance with assuming constant effective mass and effective mass depending on material (fig. \ref{fig:single_barier}).

\begin{figure}[h]
    \begin{subfigure}{0.49\textwidth}
        \begin{center}
            \includegraphics[width=0.95\textwidth]{../plots/single_barrier.pdf}
        \end{center}
        \caption{}
    \end{subfigure}
    \begin{subfigure}{0.49\textwidth}
        \begin{center}
            \includegraphics[width=0.95\textwidth]{../plots/single_barrier_const_mass.pdf}
        \end{center}
        \caption{}
    \end{subfigure}

    \caption{Transmittance and reflectance of electron passing through the barrier with (a) effective mass dependent on material (b) constant effective mass.}
    \label{fig:single_barier}
\end{figure}

Changing effective mass of electron depending on material changes values of transmittance and reflectance.
Firstly we can see that first maximum of transmittance is slightly moved.
Electron needs less energy to pass through barrier with nearly 100\% transmittance for case in changing electron mass.
But then in both cases we see loss of transmittance which comes from electron oscillations inside barrier.
This loss is bigger in changing mass case.

\newpage

\section*{Task 2}

In second task we performed calculations for double barrier.
Transmittance and reflectance were calculated (fig. \ref{fig:double_barrier}).

\begin{figure}[h]
    \begin{center}
        \includegraphics[width=0.6\textwidth]{../plots/double_barrier.pdf}
    \end{center}
    \caption{Transmittance and reflectance through double potential barrier.}
    \label{fig:double_barrier}
\end{figure}

A very characteristic point shows up at electron energy lower than barrier energy.
Point with transmittance equal to 1.
It is known effect of double barrier - inside potential well (between barriers) a gap is created inside which electron resonance occurs resulting high transmittance in narrow band.

In second part of this task Tsu-Esaki formula has been used to calculate current-voltage characteristics.
Before that one update to potential of device was introduced - by applying bias voltage potential is lineary decrasing (fig. \ref{fig:bias_voltage}).

\begin{figure}[h]
    \begin{center}
        \includegraphics[width=0.6\textwidth]{../plots/double_barrier_with_bias.pdf}
    \end{center}
    \caption{Potential in nanodevice after applying bias voltage 50 meV}
    \label{fig:bias_voltage}
\end{figure}

Current-voltage characteristic has been calculated for bias voltage in range $(0, 500)$ meV (fig. \ref{fig:tsu_esaki}).

\begin{figure}[ht]
    \begin{center}
        \includegraphics[width=0.6\textwidth]{../plots/iv_characteristic.pdf}
    \end{center}
    \caption{Current-voltage characteristic of double potential barrier (resonant-tunneling diode).}
    \label{fig:tsu_esaki}
\end{figure}

The system quickly achieves maximum in current, but then currect fals near to zero.
This characteristic is very similar to ones measured in laboratory.

\section*{Task 3}

In final task quantum point contact potential was created.
Potential coming from qpc has been show at fig. \ref{fig:qpc_potential}.

\begin{figure}[h]
    \begin{center}
        \includegraphics[width=0.8\textwidth]{../plots/qpc_potential_map.pdf}
    \end{center}
    \caption{Potential map in quantum point contact.}
    \label{fig:qpc_potential}
\end{figure}

Then using adiabatic approximation effective potential was calculated along $x$-dimension of qpc (fig. \ref{fig:qpc_eff_pot}).

\begin{figure}[ht]
    \begin{center}
        \includegraphics[width=0.8\textwidth]{../plots/qpc_states_energies.pdf}
    \end{center}
    \caption{Effective potential felt by electron for 5 states.}
    \label{fig:qpc_eff_pot}
\end{figure}

Using few first states (used 5) qpc conductance as a depedency of energy was calculated (fig. \ref{fig:cond}).

\begin{figure}[ht]
    \begin{center}
        \includegraphics[width=0.8\textwidth]{../plots/qpc_conductance.pdf}
    \end{center}
    \caption{Conductance as a dependency of energy in defined quantum point contact.}
    \label{fig:cond}
\end{figure}

Characteristic quantisation of conductance was observed, where one quant of conductance is equal to $2\frac{e^2}{h}$.
As the final task for two energies (50 meV and 100 meV) conductance was calculated as a function of gate voltage (fig. \ref{fig:cond_volt}).

\begin{figure}[ht]
    \begin{center}
        \includegraphics[width=0.8\textwidth]{../plots/cond_vs_voltage.pdf}
    \end{center}
    \caption{Conductance as a dependency of gate voltage in defined quantum point contact for electron energy 50 meV and 100 meV.}
    \label{fig:cond_volt}
\end{figure}

Conductance is also quantized versus plot voltage.
Conductance at 0 voltage is equal to five which is chosen number of states used in calculations.
For bigger number of used states bigger value of conductance would be calculated at 0 energy.

\section*{Summary}

Transfer matrix method has been show to be very accurate in one dimensional problems.
We can use Tsu-Esaki formula to get conductance values which are measureable.
For more dimensions we can use adiabatic approximation in order to reduce number of dimensions.

\end{document}
